% \documentclass{article}
\documentclass[accepted]{uai2023}
\usepackage[american]{babel}

\usepackage[utf8]{inputenc}
\usepackage{csquotes}

\usepackage{amsmath}
\allowdisplaybreaks
\usepackage{mathtools}
\usepackage{bm}
% \usepackage[a4paper, total={7in, 9in}]{geometry}
\usepackage{amssymb}
\usepackage{listings}
\usepackage{soul}
% \usepackage[sorting=none]{biblatex}
%     \addbibresource{references.bib}
\usepackage{natbib} % has a nice set of citation styles and commands
    \bibliographystyle{plainnat}
    \renewcommand{\bibsection}{\subsubsection*{References}}

\usepackage{hyperref}
\usepackage{algorithm}
\usepackage{algpseudocode}


\newcommand{\vect}[1]{\mathbf{#1}}
% \newcommand{\eq}[1]{Eq. \ref{#1}}
\newcommand{\px}[1]{\cfrac{\partial #1}{\partial x}}
\newcommand{\py}[1]{\cfrac{\partial #1}{\partial y}}
\newcommand{\dx}[1]{\cfrac{\mathrm{d} #1}{\mathrm{d} x}}
\newcommand{\dy}[1]{\cfrac{\mathrm{d} #1}{\mathrm{d} y}}
\newcommand{\dt}[1]{\cfrac{\mathrm{d} #1}{\mathrm{d} t}}
\newcommand{\ds}[1]{\cfrac{\mathrm{d} #1}{\mathrm{d} s}}
\newcommand{\dnt}[2]{\cfrac{\mathrm{d}^{#1} #2}{\mathrm{d} t^{#1}}}
% \newcommand{\Err}{\mathcal{E}}
\newcommand{\Err}{\mathfrak{e}}
\newcommand{\Bound}{\mathcal{B}}
\newcommand{\Loss}{\mathrm{Loss}}
\newcommand{\Net}{\mathrm{Net}}
\renewcommand{\L}{\mathcal{L}}
\newcommand{\I}{\mathcal{I}}
\newcommand{\Int}[1]{e^{#1 t} \int_{0}^{t} e^{- #1 \tau}\mathrm{d}\tau}
\newcommand{\Intt}{\int_{0}^{t}\mathrm{d}\tau}
\renewcommand{\Re}[1]{\mathcal{R}e\left(#1\right)}


\title{Residual-Based Error Bound for Physics-Informed Neural Networks}
% \author{Shuheng Liu, Xiyue Huang, Pavlos Protopapas}
% \date{\today}
\setlength{\parindent}{0pt}
% \setlength{\parskip}{1em}

\author[1]{Shuheng Liu}
\author[2]{Xiyue Huang}
\author[3]{Pavlos Protopapas}
% Add affiliations after the authors
\affil[1, 3]{
    Institute for Applied Computational Science\\
    Harvard University\\
    Cambridge, Massachusetts, USA
}
\affil[2]{
    Data Science Institute\\
    Columbia University\\
    New York, New York, USA
}

\begin{document}
\maketitle

\begin{abstract}
    Neural networks are universal approximators and are studied for their use in solving differential equations.
    However, a major criticism lies in that there lacks estimation of error bounds for the obtained solutions.
    This paper proposes a technique to rigorously evaluate the error bound of Physics-Informed Neural Networks (PINNs) on common classes of ODEs and PDEs.
    The error bound is based purely on equation structure and equation residuals, and does not depend on assumptions of how well the networks are trained.
    The technique evaluates the error bound in an efficient manner, and can be further improved to provide tighter bounds at the cost of longer run time.
\end{abstract}

\section{Introduction}
    Differential equations are a useful mathematical tool in describing various phenomena in natural sciences, engineering, and humanity studies. 
    As universal approximators, neural networks proves to be powerful in approximating unknown functions. 
    Using back-propagation and modern computing devices, neural networks are convenient to differentiate, making them an ideal choice for solving differential equations.

    Differential equation in this work as posed w.r.t. function $v$,
    {
        \small
        \begin{equation*}
            \mathcal{D} v = f,
        \end{equation*}
    }
    where $\mathcal{D}$ is a (possibly nonlinear) differential operator and $f$ is some forcing function.
    Unlike the exact solution $v(\cdot)$, a neural network solution $u(\cdot)$ does not strictly satisfy the equation.
    Instead, it incurs an additional residual term $r$, which the network aims to minimize, to the equation, 
    {
        \small
        \begin{equation*}
            \mathcal{D} u = f + r.
        \end{equation*}
    }
    The input to $v$, $u$, $f$, and $r$ is time $t$ for ODEs and spatial coordinates $(x, y)$ for PDEs.
    We limit our reasoning to 2-dimensional PDEs in this work.
    In cases where there are multiple unknown functions, we use vectors $\vect{v}$, $\vect{u}$, and $\vect{r}$ instead of the scalar notations $v$, $u$, and $r$.

% \subsection{LOSS FUNCTION}
    The loss function of the network solution is defined as the $L^2$ norm of residual $r$ over the domain of interest,
    {
        \small
        \begin{align}
            \Loss{}(u) &:= \frac{1}{|I|} \int_{I} \|r\|^2 \mathrm{d}I = \frac{1}{|I|} \int_{I} \|\mathcal{D} u - f\|^2 \mathrm{d}I,
        \end{align}
    }
    where a spatial domain $\Omega$ is substituted for the temporal domain $I$ in the case of a PDE.

\subsection{INITIAL AND BOUNDARY CONDITIONS}\label{section:initial-and-boundary-conditions}
    In order for a neural network to satisfy initial or boundary conditions, we apply a technique named parametrization [CITATION HERE]. 
    As an intuitive example, $u(t) = (1 - e^{-t}) \Net(t) + v(0)$ guarantees that $u(t)$ satisfies the initial condition $u(0)=v(0)$. 
    The parametrization is more complicated for higher-order ODEs and most PDEs. 
    We explain the parametrization used for experiments in Section \ref{section:experiments}. 
    In this work, we assume all initial and boundary conditions are exactly satisfied.

\subsection{ERROR AND ERROR BOUND}
    The error of a neural network solution, $u$, is defined as $ \Err_u := u - v$.
    % {
    %     \small
    %     \begin{equation}\label{eq:err-definition-master}
    %         \Err_u= u - v
    %     \end{equation}
    % }
    We are interested in \textit{bounding} the error with a scalar function $\Bound$ such that
    {
        \small
        \begin{equation}
            \|\Err_u(t)\| \leq \Bound(t) \quad\text{or}\quad \|\Err_{u}(x, y)\| \leq \Bound(x, y)
        \end{equation}
    }
    where $\|\Err_u\| = \|u - v\|$ is the \textit{absolute error}.
    If $\Bound$ takes on the same value $B \in \mathbb{R}^{+}$ over the domain, it can be replaced with a constant $B$.

    Notice that multiple bounds $\Bound$ exist for the same network solution $u$.
    Our work uncovers several bounds in descending order of tightness, $\Bound^{(1)}(t) \leq \dots \leq \Bound^{(n)}(t) \leq B$. Tighter bounds incur higher computational cost, and the looser bounds (such as constant $B$) are computed faster.


\section{LITERATURE REVIEW}
    TODO

\section{EXISTING WORK}
    TODO
\section{ERROR BOUND FOR ODE}
    In this section, we consider both linear and nonlinear ODEs over the temporal domain $I=[0, T]$. 
    Initial conditions are imposed on $\frac{\mathrm{d}^k}{\mathrm{d}t^k}v(t=0)$ for $k = 0, \dots, (n - 1)$, where $n$ is the order of the ODE.

\subsection{ERROR BOUND FOR LINEAR ODE}\label{section:error-bound-for-linear-odes}
    Consider the linear ODE $\L v(t) = f(t)$, where $\L$ is a linear differential operator. 
    Its neural network solution $u$ satisfies $\L u(t) = f(t) + r(t)$. 
    With the definition of error there is
    {   
        \small
        \begin{equation} \label{eq:linear-error-master}
            \L \Err_u(t) = r(t).
        \end{equation}
    }
    With the assumption in Section \ref{section:initial-and-boundary-conditions} that $u$ satisfies the initial conditions at $t=0$, there is
    {
        \small
        \begin{equation} \label{eq:linear-error-initial-condition}
            \Err_u(0) = 0, \quad \dt{}{}\Err_u(0) = 0, \quad \dnt{2}{}\Err_u(0) = 0, \quad \dots 
        \end{equation}
    }
    With initial conditions \ref{eq:linear-error-initial-condition} known, a unique inverse transform $\L^{-1}$to $\L$ exists. 
    Applying $\L^{-1}$ to Eq. \ref{eq:linear-error-master}, there is 
    {
        \small
        \begin{equation}\label{eq:linear-error-inverse-master}
            \Err_u(t) = \L^{-1} r(t).
        \end{equation}
    }
    Hence, bounding the absolute error $\left|\Err_u\right|$ is equivalent to bounding $\left|\L^{-1} r\right|$. 
    Notice that only (a) the equation structure $\L$ and (b) the residual information $r$ are relevant to estimating the error bound. 
    All other factors, including parameters of the neural network $u$, forcing function $f$, and initial conditions, do not affect the estimation at all.

\subsubsection{Integrating Factor Technique}
    TODO: explain the goal of the next 2 subsections
\subsubsection{Single Linear ODE with Constant Coefficients}\label{section:single-linear-ode-with-constant-coefficients}
    Consider the case where $\L = \frac{\mathrm{d}^n}{\mathrm{d}t^n} + \sum_{j=0}^{n - 1} a_j \frac{\mathrm{d}^j}{\mathrm{d}t^j}$ consists of only constant coefficients $a_0, a_1, \dots, \in \mathbb{R}$.
    Its characteristic equation (defined below) can be factorized into
    {
        \small
        \begin{equation} \label{eq:single-linear-ode-characteristic-polynomial-factorization}
            \lambda^n + a_{n-1}\lambda^{n-1} + \dots + a_0 = \prod_{j=1}^{n}(\lambda - \lambda_j),
        \end{equation}
    }
    where $\lambda_1, \dots, \lambda_n \in \mathbb{C}$ are the characteristic roots. 

    It can be shown that, for a semi-stable system ($\Re{\lambda_j} \leq 0$ for all $\lambda_j$), an error bound can be formulated as
    \begin{equation} \label{eq:linear-ode-const-loose-bound}
        \left|\Err_u(t)\right| \leq \Bound_{loose}(t) := C_{\lambda_{1:n}}\, R_{\max}\, t^{Z},
    \end{equation}
    where $0\leq Z \leq n$ is the number of $\lambda_j$ whose real part is $0$, $C_{\lambda_{1:n}} := \frac{1}{Z!}\prod_{j=1; \lambda_j\neq 0}^{n} \frac{1}{\Re{-\lambda_j}}$ is a constant coefficient, and $R_{\max}:=\max_{t\in I} |r(t)|$ is the maximum absolute residual. 
    For applications where only the order of error is concerned, knowing bound \ref{eq:linear-ode-const-loose-bound} is sufficient to qualitatively estimate the error. See Alg. \ref{alg:single-linear-ode-constant-coeff-loose} for reference.

    \begin{algorithm}
        \small
        \caption{Loose Error Bound Estimation for Linear ODE with Constant Coefficients\quad (Requires Semi-Stability)}\label{alg:single-linear-ode-constant-coeff-loose}
        \textbf{Input:} Coefficients $\left\{a_j\right\}_{j=0}^{n-1}$ for operator $\L$, residual information $r(\cdot)$, domain of interest $I = [0, T]$, a sequence of time points $\left\{t_k\right\}_{k=1}^{K}$ where error bound is to be evaluated.\\
        \textbf{Output:} Error bound at given time points $\left\{\Bound(t_k)\right\}_{k=1}^{K}$.

        \begin{algorithmic}
            \Require $\L$ is stable, and $t_k \in I$ for all $k$.
            \Ensure $\left|\Err_u(t_k)\right| \leq \Bound(t_k)$ for all $k$. 
            % \vspace{0.5em}
            \State $\{\lambda_j\}_{j=1}^{n} \gets$ numerical roots of $\lambda^n+a_{n-1}\lambda^{n-1}+\dots=0$ 
            \State \textbf{assert} $\lambda_j \leq 0$ for $1 \leq j \leq n$ 
            \State $Z, C \gets 0, 1$
            \For{$j\gets 1\dots n$}
                \If{$\Re{\lambda_j} = 0$}
                    \State $Z \gets Z + 1$
                \Else
                    \State $C \gets C / \Re{-\lambda_j}$
                \EndIf
            \EndFor
            \State $C_{\lambda_{1:n}}\gets C / Z!$
            \State $R_{\max} \gets \max_{\tau \in I} |r(\tau)|$ \Comment{Use linspace with mini-steps}
            % \For{$k \gets 1 \dots K$}
                % \State $\Bound(t_k) \gets C_{\lambda_{1:n}}\, R_{\max}\, t_k^{Z} $
            % \EndFor
            \State $\left\{\Bound(t_k)\right\}_{k=1}^{K} \gets \left\{C_{\lambda_{1:n}}\, R_{\max}\, t_k^{Z}\right\}_{k=1}^{K}$
            \State \textbf{return} $\left\{\Bound(t_k)\right\}_{k=1}^{K}$
        \end{algorithmic}
        \vspace{0.5em} 
        \textbf{Note}: Polynomial roots is solvable with \cite{jenkins1970three}
    \end{algorithm}

    An issue with Eq. \ref{eq:linear-ode-const-loose-bound} and Alg. \ref{alg:single-linear-ode-constant-coeff-loose} is that it assumes $\Re{\lambda_j} \leq 0$ for all characteristic roots $\lambda_j$. 
    This is true only for stable ODEs. 
    To address this issue, we propose an alternative error bound estimation Alg. \ref{alg:single-linear-ode-constant-coeff-tight}, which runs more slowly but does not require the system to be semi-stable, and provides a tighter bound.

    Notice that bounds of $\Err_u$ in Eq. \ref{eq:linear-error-inverse-master} can be estimated if the inverse operator $\L^{-1}$ is known. 
    In the case where $\L$ only consists of constant coefficients, we can factorize its characteristic equation as Eq. \ref{eq:single-linear-ode-characteristic-polynomial-factorization}.
    Defining operator $\I_{\lambda}$ as 
    \begin{equation} \label{eq:integral-operator-definition}
        \I_\lambda \phi(t) := e^{{\lambda} t} \int_{0}^{t} e^{-{\lambda} \tau} \phi(\tau) \mathrm{d}\tau,
    \end{equation}
    we show in supplementary matrials that $\L^{-1} = \I_{\lambda_{n}} \circ \I_{\lambda_{n-1}} \circ \dots \circ \I_{\lambda_1}$ and that $\left|\I_{\lambda} \phi\right| \ \leq \I_{\Re{\lambda}} |\phi|$ for any $\lambda \in \mathbb{C}$ and function $\phi$.
    Hence, another error bound can be formulated as
    \begin{equation} \label{eq:single-linear-ode-inverse-operator-inequality}
        \Bound_{tight}(t) := \left(\I_{\Re{\lambda_{n}}} \circ \dots \circ \I_{\Re{\lambda_1}}\right) |r(t)|.
    \end{equation}
    It can be proven that $\Bound_{tight}$ in Eq. \ref{eq:single-linear-ode-inverse-operator-inequality} is tighter than $\Bound_{loose}$ when the latter is applicable,
    \begin{equation}
        \left|\Err_u(t)\right| \leq \Bound_{tight}(t) \leq \Bound_{loose}(t) \quad \forall t \in I.
    \end{equation}
    Based on Eq. \ref{eq:single-linear-ode-inverse-operator-inequality}, we propose Alg. \ref{alg:single-linear-ode-constant-coeff-tight} which computes $\Bound_{tight}$ by repeatedly evaluating integrals in \ref{eq:integral-operator-definition} using the cumulative trapezoidal rule.

    \begin{algorithm}
        \small
        \caption{Tighter Error Bound Estimation for Linear ODE with Constant Coefficients\quad  (Stable and Unstable)}\label{alg:single-linear-ode-constant-coeff-tight}
        \textbf{Input \& Output:} Same as Alg. \ref{alg:single-linear-ode-constant-coeff-loose}. 
        % \textbf{Output:} Same as Alg. \ref{alg:single-linear-ode-constant-coeff-loose}
        \begin{algorithmic}
            \Require Same as Alg. \ref{alg:single-linear-ode-constant-coeff-loose}, except $\L$ can be unstable.
            \Ensure Same as Alg. \ref{alg:single-linear-ode-constant-coeff-loose}. 
            % \vspace{0.5em}
            \State $\{\lambda_j\}_{j=1}^{n} \gets$ numerical roots of $\lambda^n+a_{n-1}\lambda^{n-1}+\dots=0$
            \State $\left\{t_\ell\right\}_{\ell=0}^{L} \gets$ linspace($0$, $T$, \normalfont{sufficient steps})
            \State $\left\{\Bound(t_\ell)\right\}_{\ell=0}^{L} \gets \left\{r(t_\ell)\right\}_{\ell=0}^{L}$
            \For{$j \gets 1 \dots n$}
                \State integral$_{\ell=0}^{L} \gets$ CumTrapz($\left\{e^{-\lambda_j t_{\ell}} \Bound(t_\ell)\right\}_{\ell=0}^{L}$, $\left\{t_\ell\right\}_{\ell=0}^{L}$) %\Comment{Approximation of $\int_{0}^{t} e^{-\lambda_j \tau} \Bound(\tau) \mathrm{d} \tau$}
                \State $\left\{\Bound(t_\ell)\right\}_{\ell=0}^{L} \gets \left\{e^{\lambda_j t_{\ell}}\cdot \text{integral}_l \right\}_{\ell=0}^{L}$ 
            \EndFor
            \State $\left\{\Bound(t_k)\right\}_{k=1}^{K} \gets $ Interpolate($\left\{\Bound(t_\ell)\right\}_{\ell=0}^{L}$, $\left\{t_\ell\right\}_{\ell=0}^{L}$, $\left\{t_k\right\}_{k=0}^{K}$) %\Comment{Interpolate using data points $\left\{t_\ell\right\}_{\ell=0}^{L}$}
            \State \textbf{return} $\left\{\Bound(t_k)\right\}_{k=1}^{K}$ 
        \end{algorithmic}

        \vspace{0.5em} 
        \textbf{Note}: CumTrapz(\,) computes cumulative values at discrete points over domain using the trapezoidal rule.
    \end{algorithm}

\subsubsection{System of Linear ODEs with Constant Coefficients} \label{section:system-of-linear-odes-with-constant-coefficients}
    Consider a system of linear ODEs with constant coefficients 
    {
        \small
        \vspace{-0.25em}
        \begin{equation}\label{eq:linear-system-master}
            \frac{\mathrm{d}}{\mathrm{d}t}\vect{v} + A\vect{v} = \vect{f}(t)
        \end{equation}
    }
    where $\vect{v}$ and $\vect{f}$ are $\mathbb{R}^n$ vectors and $A$ is a $n\times n$ matrix. Denote the Jordan canonical form of $A$ as,
    {
        \smaller
        \begingroup 
        \setlength\arraycolsep{1pt}
        \begin{equation}\label{eq:jordan-definition}
            J = P^{-1}AP= \begin{pmatrix}
                J_1 & & \\[-0.25em]
                & \ddots & \\[-0.25em]
                & & J_K
            \end{pmatrix}
            {\text{ where }}
            J_k = \begin{pmatrix}
                \lambda_k & 1\\[-0.75em]
                & \lambda_k & \ddots\\[-0.75em]
                & & \ddots & 1\\[-0.25em]
                & & & \lambda_k
            \end{pmatrix}.
        \end{equation}
        \endgroup
    }

\subsubsection{Linear ODEs with Nonconstant Coefficients}
    TODO

\subsection{NONLINEAR ODE}
    Nonlinear ODEs are hard to solve in general. 
    In this work, we only deal with nonlinear terms of the form $\varepsilon v^k(t)$, where $\varepsilon \in \mathbb{R}$ is a small number. 
    Ideally, $|\varepsilon| \ll 1$. 
    With the perturbation technique, we obtain a family of solutions $v(t;\varepsilon)$ parameterized by $\varepsilon$ at the cost of solving a (countable) collection of equations. 
    As explained below in section \ref{section:perturbation-theory}, we train finitely many networks, each approximately solves an equation in the collection.

\subsubsection{Perturbation Theory} \label{section:perturbation-theory}
    Consider the nonlinear ODE with constant coefficients
    {
        \small
        \begin{equation} \label{eq:nonlinear-ode-master}
            \L v(t) + \varepsilon v^k(t) = f(t),
        \end{equation}
    }
    where $\L$ is a linear differential operator discussed in \ref{section:error-bound-for-linear-odes} and initial conditions are specified for the system at time $t=0$. 
    Notice that each $\varepsilon \in \mathbb{R}$ corresponds to a solution $v(t; \varepsilon)$. 
    We expand the solution $v(t; \varepsilon)$ in terms of $\varepsilon$
    {   
        \small
        \begin{equation} \label{eq:nonlinear-solution-expansion}
            v(t; \varepsilon) = \sum_{j=0}^{\infty} \varepsilon^j v_j(t) = v_0(t) + \varepsilon v_1(t) + \dots
        \end{equation}
    }
    Substituting Eq. \ref{eq:nonlinear-solution-expansion} into Eq. \ref{eq:nonlinear-ode-master}, there is
    {
        \small
        \begin{gather}
            \L \sum_{j=0}^{\infty} \varepsilon^j v_j + \varepsilon \left(\sum_{j=0}^{\infty} \varepsilon^j v_j\right)^k = f \\
            % \sum_{j=0}^{\infty} \varepsilon^j \L v_j + \varepsilon \left(\sum_{j=0}^{\infty} \varepsilon^j v_j\right)^k = f \\
            \sum_{j=0}^{\infty} \varepsilon^j \L v_j + \sum_{j=0}^{\infty} \varepsilon^{j+1} \sum_{\substack{j_1+\dots+j_k = j\\j_1, \dots, j_k \geq 0}}v_{j_1}\dots v_{j_k} = f \\[-0.5em]
            \L v_0 + \sum_{j=1}^{\infty} \varepsilon^j \Bigg(\L v_j + \sum_{\substack{j_1+\dots+j_k = j - 1\\j_1, \dots, j_k \geq 0}}v_{j_1}\dots v_{j_k}\Bigg)= f \label{eq:nonlinear-equation-expansion} 
        \end{gather}
    }
    In order for Eq. \ref{eq:nonlinear-equation-expansion} to hold true for all $\varepsilon$, the coefficients for each $\varepsilon^j$ must match on both sides of Eq. \ref{eq:nonlinear-equation-expansion}. Hence,
    {
        \small
        \begin{alignat}{6}
            &\L v_0 &&= f \label{eq:expansion-epsilon-0}\\
            &\L v_1 + v_0^k &&= 0 \label{eq:expansion-epsilon-1}\\
            &\L v_2 + k v_0^{k-1}v_1 &&= 0 \label{eq:expansion-epsilon-2} \\
            &\L v_3 + \frac{k(k-1)}{2} v_0^{k-2}v_1^2 + k v_0^{k-1}v_2 &&= 0 \label{eq:expansion-epsilon-3} \\[-1em]
            &\vdots &&\phantom{=}\,\,\,\,\vdots\nonumber
        \end{alignat}
        % \vspace{-2em}
        % \begin{equation*}
        %     \vdots
        % \end{equation*}
    }

    \vspace{-1em}
    For $\varepsilon = 0$, Eq. \ref{eq:nonlinear-solution-expansion} is reduced to $v_0(t)$, which solves the linear problem $\L v=f$. 
    Also, only $v_0(t)$ is subject to the original initial conditions at $t=0$, while other components, $v_1$, $v_2$, \dots, have initial conditions of $0$ at $t=0$.

    The above system can be solved in a \textit{sequential} manner, either analytically or using neural networks,
    \begin{enumerate}
        \item Eq. \ref{eq:expansion-epsilon-0} is linear in $v_0$ and can be solved first. 
        \item With $v_0$ known, Eq. \ref{eq:expansion-epsilon-1} is linear in $v_1$ and can be solved for $v_1$. 
        \item Similarly, with $v_0$ and $v_1$ known, Eq. \ref{eq:expansion-epsilon-2} is linear in $v_2$ and can be solved for $v_2$.
        \item The process can be repeated for Eq. \ref{eq:expansion-epsilon-3} and beyond. Only a linear ODE is solved each time.
    \end{enumerate}
    To solve the system with PINNs, we approximate exact solutions $\left\{v_j(t)\right\}_{j=1}^{\infty}$ with neural network solutions $\left\{u_j(t)\right\}_{j=0}^{J}$ trained sequentially on Eq. \ref{eq:expansion-epsilon-0}, Eq. \ref{eq:expansion-epsilon-1}, and beyond. 
    In practice, we only consider components up to order $J$ to avoid the infinity in the expansion \ref{eq:nonlinear-solution-expansion}. 
    Ideally, $J$ should be large enough so that higher order residuals in expansion \ref{eq:nonlinear-solution-expansion} can be neglected.

    After obtaining $\left\{u_j(t)\right\}_{j=0}^{J}$, we can reconstruct the solution $u(t;\varepsilon) = \sum_{j=0}^{J} \varepsilon^j u_j(t)$ to the original nonlinear equation \ref{eq:nonlinear-ode-master} for varying $\varepsilon$.
    See Alg. \ref{alg:nonlinear-iterative} for details.

\subsubsection{Expansion of Bounds}
    Comparing the exact solution $v(t; \varepsilon)$ against the network solution $u(t; \varepsilon)$, the absolute error is given by 
    {
        \small
        \begin{align}
            |\Err_u(t; \varepsilon)| &= \big|u(t; \varepsilon) - v(t; \varepsilon)\big| \nonumber \\[-0.25em]
            &= \left|\sum_{j=0}^{J} \varepsilon^{j} \Big(u_j(t) - v_j(t)\Big) - \sum_{j=J+1}^{\infty} \varepsilon^j v_j(t)\right| \nonumber \\[-0.5em]
            &\leq \sum_{j=0}^{J} \Big|\Err_{uj}(t)\Big||\varepsilon|^j + \left|\sum_{j=J+1}^{\infty}\varepsilon^j v_j(t)\right| 
        \end{align}
    }
    where $\Err_{uj}(t) := u_j(t) - v_j(t)$ is the \textit{component error} between $u_j(t)$ and $v_j(t)$.
    Let $\Bound_{j}$ denote the \textit{bound component} such that $|\Err_{uj}(t)| \leq \Bound_j(t)$.
    Assuming $J$ is large and higher order terms $\left|\sum_{j=J+1}^{\infty}\varepsilon^j v_j(t)\right|$ are negligible, there is 
    {
        \small
        \begin{equation} \label{eq:nonlinear-bound-components}
            \Big|\Err_u(t; \varepsilon)\Big| \leq \Bound(t; \varepsilon) := \sum_{j=0}^{J} \Bound_j(t)\,|\varepsilon|^j 
        \end{equation}
    }
    where each bound component $\Bound_j$ can be evaluated using the techinque in Section \ref{section:error-bound-for-linear-odes}. 
    See Alg. \ref{alg:nonlinear-iterative} for details.

    \begin{algorithm}
        \small
        \caption{Iterative Method for Solution and Error Bound of Nonlinear ODE \ref{eq:nonlinear-ode-master}} \label{alg:nonlinear-iterative}
        \textbf{Input:} Linear operator $\L$, nonlinear degree $k$, domain $I=[0, T]$, highest order $J$ for expansion, a sequence of pairs $\left\{(t_\ell, \varepsilon_\ell)\right\}_{\ell=1}^{L}$ where solution $u(t; \varepsilon)$ and error bound $\Bound(t; \varepsilon)$ is to be evaluated. \\
        \textbf{Output:} Solution $\left\{u(t_\ell; \varepsilon_\ell)\right\}_{\ell=1}^{L}$ and error bound $\left\{\Bound(t_\ell; \varepsilon_\ell)\right\}_{\ell=1}^{L}$ 
        \begin{algorithmic}
            \Require $t_\ell \in I$, and $|\varepsilon_\ell|$ to be small (ideally $|\varepsilon_\ell| \ll 1$)
            \Ensure $\Err_{u}(t_\ell; \varepsilon_\ell) \leq \Bound(t_\ell; \varepsilon_\ell)$ 
            % \hspace{0.5em}

            % \State $p_0 \gets \big(\L u_0 = f, \text{subject to initial conditions}\big)$
            \State $u_0, r_0, \gets$ network solution, residual of $\L u_0 = f$
            \State $\left\{\Bound_{0}(t_\ell)\right\}_{\ell=1}^L \gets$ bound of $\left|\L^{-1}r_0\right|$ at $\left\{t_\ell\right\}_{\ell=1}^L$
            \For{$j \gets 1 \dots J$} 
                \State Define macro $\text{NL}_j[\phi]$ as $\sum_{\substack{j_1 + \dots + j_k = j-1\\ j_1, \dots, j_k \geq 0}} \phi_{j_1} \dots \phi_{j_k}$
                \State $u_j, r_j \gets$ network solution, residual of $\L u_j + \text{NL}_j[u] = 0$
                \State $\Bound_{\text{NL}} \gets \text{upper bound of }|\text{NL}_j[u] - \text{NL}_j[v]|$
                \State $\left\{\Bound_{j}(t_\ell)\right\}_{\ell=1}^L \gets$  bound of $|\L^{-1}r_j|$+$|\L^{-1}\Bound_{\text{NL}}|$ at $\left\{t_\ell\right\}_{\ell=1}^L$
            \EndFor
            \State $\left\{u(t_\ell; \varepsilon_\ell)\right\}_{\ell=1}^L \gets \left\{\sum_{j=0}^{J}\varepsilon_\ell^j u_j(t_\ell)\right\}_{\ell=1}^L $ 
            \State $\left\{\Bound(t_\ell; \varepsilon_\ell)\right\}_{\ell=1}^L \gets \left\{\sum_{j=0}^{J}\varepsilon_\ell^j \Bound_j(t_\ell)\right\}_{\ell=1}^L $ 
            \State \textbf{return} $\left\{u(t_\ell; \varepsilon_\ell)\right\}_{\ell=1}^L, \left\{\Bound(t_\ell; \varepsilon_\ell)\right\}_{\ell=1}^L$
        \end{algorithmic}
        \vspace{0.5em}

        \textbf{Note} 1: $\Bound_0$ and $\Bound_{1:J}$ can be evaluated using Alg. \ref{alg:single-linear-ode-constant-coeff-loose} or Alg. \ref{alg:single-linear-ode-constant-coeff-tight}.\\
        \textbf{Note} 2: $\Bound_\text{NL}$ can be estimated even though exact solutions $v_{0:j-1}$ are unknown. This is because $v_i \in [u_i - \Bound_i, u_i+\Bound_i]$ for all $i$, and $u_{0:j-1}$, $\Bound_{0:j-1}$ are known.
    \end{algorithm}

\section{ERROR BOUND FOR PDE}
    In this work, we consider PDEs defined on a 2-dimensional spatial domain $\Omega$ and limit our discussion to first-order linear PDEs.\footnote{Similar techniques can be used for more scenarios and higher dimensions where the method of characteristics is applicable.} 
    Consider the first-order linear PDE,
    { 
        \small
        \begin{equation}\label{eq:pde-master}
            a(x, y) \partial_x v + b(x, y) \partial_y v + c(x, y)v = f(x, y)
        \end{equation}
    }
    with Dirichlet boundary constraints defined on $\Gamma \subset \partial \Omega$,
    {
        \small
        \begin{equation}\label{eq:pde-bc-master}
            v\big|_{(x, y) \in \Gamma} = g(x, y),
        \end{equation}
    }
    where $a$, $b$, and $c$ are locally Lipschitz on $\overline\Omega$. For reference, continuous differentiability implies local-Lipschitzness, which implies continuity.

    We partition the domain into infinitely many characteristic curves $\mathcal{C}$, each passing through a point $(x_0, y_0) \in \Gamma$. The resulting curve is a parameterized integral curve 
    {
        \small
        \begin{equation*} 
            \mathcal{C}: \begin{cases*}
                x'(s) = a(x, y) \\
                y'(s) = b(x, y) 
            \end{cases*} 
            \quad
            \text{where}
            \,\,
            (\cdot)' = \ds{}
            \quad
            \text{and} 
            \quad
            \begin{aligned}
                x(0) &= x_0 \\
                y(0) &= y_0.
            \end{aligned}
        \end{equation*}
    }
    % Note that the system \ref{eq:parameter-eq-differential} can be nonlinear but needs not always be solved for a loose error bound to be evaluated. 
    % Still, knowing the exact characteristic curves $\mathcal{C}$ leads to a tighter bound.
    For any $(x(s), y(s))$ on $\mathcal{C}$, functions $(v, a, b, c, f)$ can be viewed as univariate functions of $s$. By chain rule, there is
    {
        \small
        \begin{equation*}
            a(x, y)\partial_x v + b(x, y)\partial_y v = x'(s)\partial_x v  + y'(s)\partial_y v = v'
        \end{equation*}
    }
    Hence, the original PDE can be reformulated as an ODE
    {
        \small
        \begin{equation}
            v' + c(s)v = f(s) \quad \text{s.t.} \quad v(0) = g(x_0, y_0)
        \end{equation}
    }

    In particular, if $c(x, y) \neq 0$ for all $(x, y) \in \Omega$, both sides of Eq. \ref{eq:pde-master} can be divided by $c(x, y)$, resulting in a residual of $r(x, y)/c(x, y)$ where $r(x, y)$ is the residual of the original problem. By Eq. \ref{eq:linear-ode-const-loose-bound}, a constant error bound on $\mathcal{C}$ is $|\Err_u(s)| \leq \max_{s}\left|r(s)/c(s)\right|$. Hence, a trivial constant error bound $B$ (see Alg. \ref{alg:linear-first-order-pde-constant}) over domain $\Omega$ is
    {
        \small
        \begin{equation}
            |\Err_u(x, y)| \leq B :=\max_{(x, y)\in \Omega}\left|\frac{r(x, y)}{c(x, y)}\right|.
        \end{equation}
    }

    \begin{algorithm}
        \small
        \caption{Constant Err Bound for Linear 1st-Order PDE}\label{alg:linear-first-order-pde-constant}
        \textbf{Input:} Coefficient $c(x, y)$ in Eq. \ref{eq:pde-master}, residual information $r(x, y)$, domain of interest $\Omega$\\
        \textbf{Output:} A constant error bound $B \in \mathbb{R}^+$
        \begin{algorithmic}
            \Require $c(x, y) \neq 0$ for all $(x, y) \in \Omega$
            \Ensure $|\Err_{u}(x, y)| \leq B$ for all $(x, y) \in \Omega$

            % \vspace{0.5em}
            \State $\left\{(x_k, y_k)\right\}_{k} \gets$ sufficiently dense mesh grid over $\Omega$
            \State $\displaystyle B \gets \max_{k} \left| \frac{r(x_k, y_k)}{c(x_k, y_k)}\right|$
            \vspace{-0.25em}
            \State \textbf{return} $B$
        \end{algorithmic}
    \end{algorithm}

    Independent of the assumption $c(x, y)\neq 0$, in scenarios where the curve $\mathcal{C}$ passing through any $(x, y)$ can be computed, the error can be computed using Alg. \ref{alg:linear-first-order-pde-general}

    \begin{algorithm}
        \small
        \caption{General Err Bound for Linear 1st-Order PDE}\label{alg:linear-first-order-pde-general}
        \textbf{Input:} Coefficients $a$, $b$, $c$ in Eq. \ref{eq:pde-master}, residual information $r(x, y)$, domain of interest $\Omega$, Dirichlet boundary $\Gamma\subset \partial \Omega$, a sequence of points $\left\{(x_\ell, y_\ell)\right\}_{\ell=1}^{L}$ where error is to be bounded\\
        \textbf{Output:} Error bound $\left\{\Bound(x_\ell, y_\ell)\right\}_{\ell=1}^{L}$ at given points
        \begin{algorithmic}
            \Require Integral curve of the vector field $\big[a(x, y)\,\, b(x, y)\big]^T$ passing through any point $(x_\ell, y_\ell) \in \Omega$ is solvable
            \Ensure $|\Err_{u}(x_\ell, y_\ell)| \leq \Bound(x_\ell, y_\ell)$ for all $\ell$

            % \vspace{0.5em} % no need to skip since the \begin{cases}\end{cases} takes vertical space
            \State $\mathcal{C}_{\text{gen}} \gets $ general solution (integral curves) to {\smaller $\begin{cases}x'(s) = a(x, y) \\ y'(s) = b(x, y)\end{cases}$}
            \vspace{-1em} % move this line up a bit
            \For{$\ell \gets 1 \dots L$}
                \State $\mathcal{C}:(x(s), y(s))\gets$ instance of $\mathcal{C}_{\text{gen}}$ passing through $(x_\ell, y_\ell)$
                \State $s^* \gets$ solution to $x(s) = x_\ell,\, y(s)=y_\ell$
                \vspace{-0.25em}
                \State $\displaystyle \Bound(x_\ell, y_\ell) \gets e^{c(s^*)}\int_{0}^{s^*}r(x(s), y(s)) e^{-c(x(s),y(s))\,s}\mathrm{d}s$ 
                \vspace{-0.5em}
            \EndFor
            \State \textbf{return} $\left\{\Bound(x_\ell, y_\ell)\right\}_{\ell=1}^{L}$
        \end{algorithmic}
    \end{algorithm}
% \subsection{Hyperbolic PDEs}
%     TODO

\section{RELEVANT EXPERIMENTS}\label{section:experiments}
    TODO

\subsection{HIGHER DIMENSIONAL SYSTEM WITH CONSTANT COEFFICIENTS} \label{section:high-dimension}
    TODO

\subsection{LINEAR ODE SYSTEM WITH NONCONSTANT COEFFICIENTS -- NONHARMONIC OSCILLATOR} \label{section:experiment-nonharmonic-oscillator}
    TODO

\subsection{NONLINEAR ODE -- DUFFING EQUATION} \label{section:experiment-duffing}
    TODO

\subsection{LINEAR PDE SYSTEM WITH NONCONSTANT COEFFICIENTS } \label{section:experiment-attractor}
    TODO

\section{FUTURE WORK}
    TODO

\bibliography{references}


\end{document}
